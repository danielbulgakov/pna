% =============================================================================
% Compile parameters
% =============================================================================

\documentclass[12pt, russian]{extarticle}
\usepackage[T2A]{fontenc}
\usepackage{fontspec}
\usepackage{amsmath,amsthm}
\defaultfontfeatures{Ligatures={TeX},Renderer=Basic}
\setmainfont[Ligatures={TeX,Historic}]{Times New Roman}
\usepackage[a4paper,
left=25mm,
right=15mm,
top=20mm,
bottom=20mm]{geometry}
\usepackage[linktoc=all]{hyperref}
\usepackage{titlesec}
\titlelabel{\thetitle.\quad}
\usepackage{tocloft}

% Remove \bfseries from section titles in ToC
\renewcommand{\cftsecfont}{}
% Remove \bfseries from section titles' page in ToC
\renewcommand{\cftsecpagefont}{}
\renewcommand{\cftsecaftersnum}{.}
\usepackage{titlesec}

% Change font size for types
\titleformat*{\section}{\large\bfseries}
\titleformat*{\subsection}{\large\bfseries}
\titleformat*{\subsubsection}{\large\bfseries}

\setlength{\parindent}{1.25cm}
\setlength{\parskip}{0.4cm}
\font\subtitlefont=cmr12 at 12pt
\font\titlefont=cmr12 at 24pt
\usepackage{color}
\usepackage{mathtools}
\usepackage{listings}
\usepackage{graphicx}
\usepackage{tocloft}
\usepackage{indentfirst}
\usepackage{enumitem}
\usepackage{graphicx}
\usepackage{subcaption}
\usepackage[russian]{babel}
\usepackage{setspace}
\setlength{\marginparwidth}{2cm}
\usepackage{todonotes}
\usepackage{tikz}
\usepackage{pdfpages}
\usetikzlibrary{shapes, arrows, positioning}
\renewcommand{\contentsname}{}
\renewcommand{\cftsecleader}{\cftdotfill{\cftdotsep}}
\graphicspath{ {./resources/} }

\usepackage{listings}
\usepackage{float}

\lstset{
    numbers=left,                   % where to put the line-numbers
    numberstyle=\small \ttfamily \color[rgb]{0.4,0.4,0.4},
                % style used for the linenumbers
    showspaces=false,               % show spaces adding special underscores
    showstringspaces=false,         % underline spaces within strings
    showtabs=false,                 % show tabs within strings adding particular underscores
    frame=lines,                    % add a frame around the code
    tabsize=4,                        % default tabsize: 4 spaces
    breaklines=true,                % automatic line breaking
    breakatwhitespace=false,        % automatic breaks should only happen at whitespace
    basicstyle=\ttfamily,
    %identifierstyle=\color[rgb]{0.3,0.133,0.133},   % colors in variables and function names, if desired.
    keywordstyle=\color[rgb]{0.133,0.133,0.6},
    commentstyle=\color[rgb]{0.133,0.545,0.133},
    stringstyle=\color[rgb]{0.627,0.126,0.941},
}

% =============================================================================
% Useful 
% =============================================================================

% insert code
% \lstinputlisting[language=Python]{./resources/classification.py}


% insert image
% \begin{figure}[H]
%     \centering
%     \includegraphics[scale=0.6]{resources/2.png}
% \end{figure}

% =============================================================================
% End of compile parameters
% =============================================================================

\title{}
\author{}
\date{}

\begin{document}

% =============================================================================
% Global titlepage
% =============================================================================

\begin{titlepage}

    \begin{center}
        МИНИСТЕРСТВО НАУКИ И ВЫСШЕГО ОБРАЗОВАНИЯ РОССИЙСКОЙ ФЕДЕРАЦИИ \\
        Федеральное государственное автономное образовательное учреждение \\
        высшего образования \\
        \textbf{
            «Национальный исследовательский \\
            Нижегородский государственный университет им. Н.И. Лобачевского»\\ (ННГУ)
        }
    \bigbreak

    \vspace{2em}
        \textbf{
            Институт информационных технологий, математики и механики
            \bigbreak
            Кафедра: математического обеспечения и суперкомпьютерных технологий
        }

        Направление подготовки: «Программная инженерия» \\
        Профиль подготовки: «Технологии цифровой трансформации»

        \bigbreak
        \bigbreak
        \bigbreak

        \textbf{ОТЧЕТ} \\
        по предмету \\
        \textbf{
        ``Параллельные численные методы''}
        \bigbreak

    \end{center}

    \vspace{5em}

    \begin{flushright}
        {\bfseries Выполнил:} студент группы \\ 3824М1ПР1 Д.Э. Булгаков\\
        \hfill Подпись \hspace{5em} \newline \\
        {\bfseries Проверил:} д.т.н., доц.,\\ К.А.Баркалов \\
        \hfill Подпись \hspace{5em} \newline \\
    \end{flushright}


    \vspace{\fill}

    \begin{center}
        Нижний Новгород\\2025
    \end{center}

\end{titlepage}

% =============================================================================
% Main content
% =============================================================================

% ========== Set global spacing ==========
\begin{spacing}{1.5}

% ========== Table of content ==========
\tableofcontents
\thispagestyle{empty}
\newpage

% Params to make the following text start with
% its page number
\pagestyle{plain}
\setcounter{page}{3}


\section{Введение}

Разложе́ние Холе́цкого (метод квадратного корня) — представление симметричной положительно определённой матрицы \( A \) в виде
\[
A = LL^{T},
\]
где \( L \) — нижняя треугольная матрица со строго положительными элементами на диагонали. Иногда разложение записывается в эквивалентной форме:
\[
A = U^{T}U,
\]
где \( U = L^{T} \) — верхняя треугольная матрица.

Разложение Холецкого всегда существует и единственно для любой симметричной положительно определённой матрицы.

\newpage
\section{Постановка задачи}

Необходимо реализовать блочное разложение Холецкого для симметричной положительно определённой матрицы \( A \), то есть представить её в виде произведения:
\[
A = LL^{T},
\]
где \( L \) — нижняя треугольная матрица со строго положительными элементами на диагонали.

Требуется использовать технологии параллельного программирования с помощью OpenMP для повышения производительности.


\newpage
\section{Последовательный алгоритм}

\subsection*{Шаги алгоритма}

Пусть матрица \( A \) имеет размерность \( n \times n \), и необходимо найти нижнюю треугольную матрицу \( L \), для чего выполняются следующие шаги:

\begin{enumerate}
\item
\textbf{Инициализация:} Создаём матрицу \( L \) размерности \( n \times n \), инициализируя все её элементы нулями. Элементы на диагонали будут вычисляться как положительные числа, а элементы под диагональю — как значения, соответствующие разложению.


\item
\textbf{Цикл по строкам и столбцам:} Основной цикл состоит из двух вложенных:
   \begin{itemize}
       \item Внешний цикл перебирает строки матрицы \( A \) и вычисляет элементы матрицы \( L \).
       \item Внутренний цикл вычисляет элементы \( L \) в каждой строке, используя формулы разложения Холецкого.
   \end{itemize}
   
   Для каждого элемента \( L_{i,j} \), где \( i \geq j \), вычисление происходит по следующей формуле:
   \[
   L_{i,j} = \sqrt{A_{i,j} - \sum_{k=1}^{j-1} L_{i,k} L_{j,k}},
   \]
   если \( i = j \) (элементы на диагонали), или
   \[
   L_{i,j} = \frac{1}{L_{j,j}} \left( A_{i,j} - \sum_{k=1}^{j-1} L_{i,k} L_{j,k} \right),
   \]
   если \( i > j \) (элементы ниже диагонали).

\item
\textbf{Построение матрицы \( L \):} После вычисления всех элементов матрицы \( L \), мы получаем её как результат разложения Холецкого. После этого матрица \( L \) может быть использована для проверки, путём вычисления \( LL^T \), и сравнения с исходной матрицей \( A \).

\item
\textbf{Завершение:} Алгоритм завершается после вычисления всех элементов матрицы \( L \).

\end{enumerate}

\subsection*{Время выполнения}

Время выполнения этого алгоритма в последовательной версии имеет сложность \( O(n^3) \), что обусловлено необходимостью вычисления каждого элемента матрицы \( L \) через сумму по всем предыдущим элементам строки и столбца. Это приводит к тройному вложенному циклу, где каждый цикл выполняется \( n \) раз.

\[
T_{\text{seq}} = O(n^3).
\]

\newpage
\section{Параллельный алгоритм}

\section*{Параллельный алгоритм разложения Холецкого}

Для распараллеливания алгоритма разложения Холецкого  применяется блочное разложение.

\subsection*{Алгоритм}

\begin{enumerate}
\item
\textbf{Инициализация матрицы \( L \)}. На первом шаге происходит инициализация всех элементов матрицы \( L \) значениями нулей. Это выполняется параллельно для всех элементов матрицы:
\begin{verbatim}
#pragma omp parallel for collapse(2)
for (int i = 0; i < n; ++i)
    for (int j = 0; j < n; ++j)
        L[i * n + j] = 0.0;
\end{verbatim}

\item
\textbf{Обработка блоков матрицы}. Основной цикл алгоритма выполняет обработку матрицы по блокам. Каждый блок размером \( B \times B \) обрабатывается отдельно, начиная с диагональных элементов:
\begin{verbatim}
for (int k = 0; k < n; k += blockSize) {
    int end = std::min(k + blockSize, n);
\end{verbatim}

\begin{enumerate}
\item 
\textbf{Вычисление диагональных элементов} для каждого блока:
\begin{verbatim}
for (int i = k; i < end; ++i) {
    double sum = 0.0;
    for (int p = 0; p < i; ++p)
        sum += L[i * n + p] * L[i * n + p];
    L[i * n + i] = std::sqrt(A[i * n + i] - sum);
}
\end{verbatim}

\item
\textbf{Вычисление элементов нижней треугольной части блока}:
\begin{verbatim}
for (int j = i + 1; j < end; ++j) {
    double sum = 0.0;
    for (int p = 0; p < i; ++p)
        sum += L[j * n + p] * L[i * n + p];
    L[j * n + i] = (A[j * n + i] - sum) / L[i * n + i];
}
\end{verbatim}

\item
\textbf{Обновление оставшихся блоков}:
Обновление правого блока (верхняя часть матрицы \( A \), которая не была обработана в первом цикле):
\begin{verbatim}
#pragma omp parallel for
for (int i = end; i < n; ++i) {
    for (int j = k; j < end; ++j) {
        double sum = 0.0;
        for (int p = 0; p < j; ++p)
            sum += L[i * n + p] * L[j * n + p];
        L[i * n + j] = (A[i * n + j] - sum) / L[j * n + j];
    }
}
\end{verbatim}

\end{enumerate}

\end{enumerate}

\subsection*{Оценка сложности}

Параллельный алгоритм имеет теоретическую сложность \( O\left(\frac{n^3}{p}\right) \), где \( p \) — количество потоков, используемых в OpenMP. Эта оценка предполагает идеальную масштабируемость, что, однако, редко наблюдается на практике из-за накладных расходов на создание и синхронизацию потоков.


\newpage
\section{Результаты}

Алгоритм был реализован на языке C++ с использованием библиотеки OpenMP для параллельных вычислений. Для проверки корректности параллельного алгоритма была использована случайно сгенерированная положительно определенная матрица. Результаты работы программы для различных размеров матриц и количества потоков приведены ниже.

Далее для тестов использовались случайно сгенерированные матрицы. Размер блока выбран 32. Время выполнения в секундах.

\begin{table}[h!]
\centering
\begin{tabular}{|c|c|c|c|c|}
\hline
\textbf{Потоки/Размерность} & \textbf{1000x1000} & \textbf{3000x3000} & \textbf{5000x5000} \\
\hline
1 & 0.4551 & 12.1241 & 56.6328 \\
\hline
2 & 0.2386 & 6.2803 & 28.8508 \\
\hline
4 & 0.1358 & 3.2617 & 14.8578 \\
\hline
6 & 0.0994 & 2.3387 & 10.5903 \\
\hline
\end{tabular}
\caption{Время выполнения разложения Холецкого для разных размеров матриц и количества потоков}
\label{tab:results}
\end{table}


\newpage
\section{Заключение}

В ходе выполнения работы был изучен алгоритм разложения Холецкого. На основе полученных знаний был реализован блочный алгоритм разложения Холецкого на языке C++ с использованием библиотеки OpenMP для параллельных вычислений.

\newpage
\section{Список литературы}

\begin{enumerate}
    \item Нестеренко, В. П., Рябов, А. В. \textit{Методы численного решения линейных систем.} – М.: Наука, 2009. – 512 с.
    \item Чесноков, В. И. \textit{Параллельные вычисления. Теория и практика.} – М.: МЦНМО, 2017. – 304 с.
    \item Папалексис, В. В., Ортега, Дж. \textit{Введение в параллельные методы решения линейных систем.} — М.: Мир, 1991. — 376 с.
    \item OpenMP Architecture Review Board. \textit{OpenMP Application Programming Interface. Version 4.5.} – 2015. – Режим доступа:. – Загл. с экрана.
\end{enumerate}


\newpage
\section{Приложение}

\begin{verbatim}
void Cholesky_Decomposition(double* A, double* L, int n) {
    int blockSize = 32;

    #pragma omp parallel for collapse(2)
    for (int i = 0; i < n; ++i)
        for (int j = 0; j < n; ++j)
            L[i * n + j] = 0.0;

    for (int k = 0; k < n; k += blockSize) {
        int end = std::min(k + blockSize, n);

        for (int i = k; i < end; ++i) {
            double sum = 0.0;
            for (int p = 0; p < i; ++p)
                sum += L[i * n + p] * L[i * n + p];
            L[i * n + i] = std::sqrt(A[i * n + i] - sum);

            for (int j = i + 1; j < end; ++j) {
                sum = 0.0;
                for (int p = 0; p < i; ++p)
                    sum += L[j * n + p] * L[i * n + p];
                L[j * n + i] = (A[j * n + i] - sum) / L[i * n + i];
            }
        }

        #pragma omp parallel for collapse(2)
        for (int i = end; i < n; ++i) {
            for (int j = k; j < end; ++j) {
                double sum = 0.0;
                for (int p = 0; p < j; ++p)
                    sum += L[i * n + p] * L[j * n + p];
                L[i * n + j] = (A[i * n + j] - sum) / L[j * n + j];
            }
        }
    }
}
\end{verbatim}

\end{spacing}
\end{document}